\documentclass[11pt,a4paper]{article}

\usepackage[slovene]{babel}
\usepackage[utf8x]{inputenc}
\usepackage{graphicx}
\usepackage{url}
\pagestyle{plain}

\begin{document}
\title{Poročilo pri predmetu \\
Analiza podatkov s programom R}
\author{Neža Ahčin}
\maketitle

\section{Izbira teme}

Tema mojega projekta je doping v športu. Osnovna ideja projekta je analiza podatkov in različnih vrst spremenljivk, kot so na primer: kako se je število dopingiranih športnikov v preteklosti spreminjalo, iz kje so prihajali, v katerih športih so se udejstvovali, katera tekmovanja (Olimpijske igre, svetovna prvenstva, Tour de France) so postregla z največ dopinškimi škandali, s katerimi nedovoljenimi sredstvi si športniki napogosteje pomagajo do čim boljšega rezultata.



Podatke sem oziroma bom pridobila iz naslednjih spletnih virov:

\begin{itemize} 
\item \url{http://en.wikipedia.org/wiki/Doping_at_the_Olympic_Games}
\item \url{http://en.wikipedia.org/wiki/List_of_doping_cases_in_athletics}
\item \url{http://sportsanddrugs.procon.org/view.resource.php?resourceID=004420}
\item \url{http://en.wikipedia.org/wiki/List_of_doping_cases_in_sport}
\item \url{www.huffingtonpost.com/2014/02/16/olympics-drug-testing-medals-stripped_n_4789565.html}
\item \url{http://en.wikipedia.org/wiki/List_of_stripped_Olympic_medals}
\end{itemize}

Cilj projekta je spoznati program R skozi analizo in obdevalo podtakov, predvsem pa pridobiti pozitivno oceno pri predmetu ANPP.

\section{Obdelava, uvoz in čiščenje podatkov}

V svoj program sem uvozila 3 tabele CSV oblike. Eno sem našla na spletu, preostali dve sem v Excellu napisala sama s pomočjo različnih spletnih virov, ju pretvorila v CSV obliko ter ju nato uvzila v projekt. Ostale 3 tabele sem naredila s preoblikovanjem html tabele, ki sem jo uvzila s spletne strani. Ker so se mi pri uvozu html tabele nekateri stolpci med seboj pomešali, sem iz ene tabele naredila tri manjše podtabele, s tem pa sem tudi olajšala nadaljno obdelava podatkov. V vsaki tabeli sem preverila tip spremenljivk, ga po potrebi spremenila, v eni izmed tabel pa sem tudi dodala novo urejenostno spremenljivko. Odstranila sem nepotrebne znake, vrstice, stolpce in dodala nove. 

Na podlagi danih podatkov zbranih v tabelah sem narisala 8 grafov (5 stolpičnih, 2 tortna diagrama ter 1 graf).
Program za risnje grafov sem napisala v datoteki grafi.r, ki sem ga shranila v mapo slike. Zaradi lažjega risanja grafov, sem na novo definirala nekaj funkcij, dodala novo tabelo oziroma novo spremenljivko.
Na koncu pa sem vse grafe še izvozila v PDF datoteki.

\includepdf[pages={1-8}]{../slike/grafi.pdf}

\section{Analiza in vizualizacija podatkov}

\includegraphics{../slike/povprecna_druzina.pdf}

\section{Napredna analiza podatkov}

\includegraphics{../slike/naselja.pdf}

\end{document}
