\documentclass[11pt,a4paper]{article}

\usepackage[slovene]{babel}
\usepackage[utf8x]{inputenc}
\usepackage{graphicx}
\usepackage{url}
\usepackage{pdfpages}
\usepackage{amsmath}
\usepackage{hyperref}

\pagestyle{plain}

\begin{document}
\title{Poročilo pri predmetu \\
Analiza podatkov s programom R}
\author{Neža Ahčin}
\maketitle
\newpage
\section{Izbira teme}

Tema mojega projekta je doping v športu. Osnovna ideja projekta je analiza podatkov in različnih vrst spremenljivk, kot so na primer: kako se je število dopingiranih športnikov v preteklosti spreminjalo, iz kje so prihajali, v katerih športih so se udejstvovali, katera tekmovanja (Olimpijske igre, svetovna prvenstva, Tour de France) so postregla z največ dopinškimi škandali, s katerimi nedovoljenimi sredstvi si športniki napogosteje pomagajo do čim boljšega rezultata.

Podatke sem oziroma bom pridobila iz naslednjih spletnih virov:
 
\begin{itemize} 
\item \url{http://en.wikipedia.org/wiki/Doping_at_the_Olympic_Games}
\item \url{http://en.wikipedia.org/wiki/List_of_doping_cases_in_athletics}
\item \url{http://sportsanddrugs.procon.org/view.resource.php?resourceID=004420}
\item \url{http://en.wikipedia.org/wiki/List_of_doping_cases_in_sport}
\item \url{www.huffingtonpost.com/2014/02/16/olympics-drug-testing-medals-stripped_n_4789565.html}
\item \url{http://en.wikipedia.org/wiki/List_of_stripped_Olympic_medals}\\
\end{itemize}
Cilj projekta je spoznati program R skozi analizo in obdevalo podtakov, predvsem pa pridobiti pozitivno oceno pri predmetu ANPP.

\newpage

\section{Obdelava, uvoz in čiščenje podatkov}

V svoj program sem uvozila 3 tabele CSV oblike. Eno sem našla na spletu, preostali dve sem v Excellu napisala sama s pomočjo različnih spletnih virov, ju pretvorila v CSV obliko ter ju nato uvzila v projekt. Ostale 3 tabele sem naredila s preoblikovanjem html tabele, ki sem jo uvzila s spletne strani. Ker so se mi pri uvozu html tabele nekateri stolpci med seboj pomešali, sem iz ene tabele naredila tri manjše podtabele, s tem pa sem tudi olajšala nadaljno obdelava podatkov. V vsaki tabeli sem preverila tip spremenljivk, ga po potrebi spremenila, v eni izmed tabel pa sem tudi dodala novo urejenostno spremenljivko. Odstranila sem nepotrebne znake, vrstice, stolpce in dodala nove. \\
\\
Na podlagi danih podatkov zbranih v tabelah sem narisala 8 grafov (5 stolpičnih, 2 tortna diagrama ter 1 graf).
Program za risnje grafov sem napisala v datoteki grafi.r, ki sem ga shranila v mapo slike. Zaradi lažjega risanja grafov, sem na novo definirala nekaj funkcij, dodala novo tabelo oziroma novo spremenljivko.
Na koncu pa sem vse grafe še izvozila v PDF datoteki.


%\includepdf[pages={1-4}]{../slike/grafi.pdf}


\newpage

\section{Analiza in vizualizacija podatkov}

V tretji fazi sem podatke, ki sem jih uvozila v prejšnji fazi prikazala na zemljevidu.
V ta namen sem v R uvozila zemljevid sveta v obliki SHP.
Prva težava na katero sem naletela v tej fazi, je bila ta, da dveh držav, za kateri sem uvozila podatke ni bilo na zemljevidu, tako da ju nisem mogla prikazati. 
Preden sem se lotila risanja zemljevidov sem uredila podatke, tako da so se že prej uvoženi podatki ujemali z državami na zemljevidu sveta.za ujemanje držav je bilo nekatere podatke potrebno tudi spremeniti oziroma nekoliko popraviti (npr.: Great Britain - United Kingdom, West Gemrany - Germany Soviet Union - Russia)\\
\\

Narisala sem 4 zemljevide:
\begin{itemize}

\item{Zemljevid1 prikazuje število dopingiranih športnikov iz posamezne države.}


\item{Zemljevid2 prikazuje katero je najpogosteje uporabljeno nedovoljeno sredstvo v posamezni državi (imenska spremenljivka).
Ker je moja baza podtakov relativno majhna, sem pri risanju tega zemljevida upoštevala zgolj države, v katerih je bilo vsaj 6 dopinških primerov. Podatke za ostale države sem ocenila za nerelavantne.}

\item{Zemljevid3 prikazuje število odvzetih medalj na Olimpijskih igrah, zaradi pozitivnega doping testa}

\item{Zemljevid4 prikazuj povprečno število dopinških primerov glede na državo, ki je gostila Olimpijske igre. 
Ker so nekatere države gostile Olimpijske igre večkrat, sem se odločila da bom vzela povprečje, s tem pa so tudi dani podatki postali bolj primerljivi.}\\
\end{itemize}
Na zemljevide sem dodala nekaj oznak, vendar z njimi raje nisem pretiravala, saj bi s tem zemljevidi postali težje berljivi.\\
\\
Program za risanje zemljevidov sem napisala v datoteki vizualizacija.r, ki se nahaja v mapi vizualizacija. 
Na koncu pa sem vse zemljevide še izvozila v PDF obliki in jih shranila v mapo slike.

%\includegraphics[width=\textwidth]{../slike/zemljevidi.pdf}

\section{Napredna analiza podatkov}

%\includegraphics{../slike/naselja.pdf}

\end{document}
