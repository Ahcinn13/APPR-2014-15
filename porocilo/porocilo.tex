\documentclass[11pt,a4paper]{article}

\usepackage[slovene]{babel}
\usepackage[utf8x]{inputenc}
\usepackage{graphicx}
\usepackage{url}
\usepackage{pdfpages}
\usepackage{amsmath}
\usepackage{hyperref}

\pagestyle{plain}

\begin{document}
\title{Poročilo pri predmetu \\
Analiza podatkov s programom R}
\author{Neža Ahčin}
\maketitle
\newpage
\section{Izbira teme}

V zadnjih letih smo bili priča, kar nekaj odmevnim dopinškim škandalom. V medijih so se pojavljala predvsem imena športnikov, kot so Lance Armstrong, Marion Jones, Asafa Powell, Tayson Gay, vendar pa vemo, da za tako organizirano dejavnostjo, kot je doping, stoji mnogo večje število ljudi: od zdravnikov, trenerjev, managerjev, sotekmovalcev do celotne ekipe in navsezadnje družine. Ker je doping postal eden izmed največjih problemov v svetu športa, sem za naslov svojega projekta izbrala doping v športu.
\newline
Osnovna ideja projekta je analiza podatkov in različnih vrst spremenljivk, kot so na primer: kako se je število dopingiranih športnikov v preteklosti spreminjalo, iz kje so prihajali, katere športne panoge so postregla z največ dopinškimi škandali, s katerimi nedovoljenimi sredstvi si športniki napogosteje pomagajo do čim boljšega rezultata...


Doping je ena iz med najbolj ražširjenih oblik korupcije v športu, zato so temu primerno mnogi podatki tudi prirejeni in zamolčani. Na podlagi tega sem se odločila, da se pri svojem projektu osredotočim predvsem na doping na Olimpijskih igrah. Olimpijske igre slovijo kot šprtna prireditev, ki poteka v duhu Fair play, zato se mi ti podatki zdijo tud najbolj relevantn.


Cilj projekta je spoznati program R skozi analizo in obdevalo podtakov, predvsem pa pridobiti pozitivno oceno pri predmetu ANPP.

\newpage

\section{Obdelava, uvoz in čiščenje podatkov}

Program za obdelavo, uvoz in čiščenje podatkov je napisan v datotetki \verb|uvoz.r|, ki je shranjena v mapi \verb|uvoz|.
\\
\\
V svoj program sem uvozila naslednje table:
\\
\\
Tabela \verb|doping.data|:
\\
V program sem jo uvzila kot CSV datoteko. Vsebuje podatke o številu dopinških testov in pozitivnih vzorcih.
Sestavljena je iz 297 opazovanj in 5 spremenljivk
\begin{enumerate}
\item{\verb|Year| (številska spremenljivka)}
\item{\verb|Sport| (imenska spremenljivka)}
\item{\verb|Samples| (številska spremenljivka)}
\item{\verb|Total.findings| (številska spremenljivka)}
\item{\verb|procent.ofDopingCases| (številska spremenljivka)}

\end{enumerate}

\
\\
Na podlagi tabele \verb|doping.data| sem sestavila tabelo \verb|dopingBySports|, ki je sestavljena iz 39 opazovanj in 3 spremeljivk:
\begin{enumerate}

\item{\verb|Sport| (imenska spremenljivka)}
\item{\verb|Average| (številska spremenljivka)}
\item{\verb|Number.ofDopingCases| (urejenostna spremenljivka)}


\end{enumerate}

\
\\
Tabela \verb|doping.POI|:
V program sem jo uvozila kot CSV datoteko. V tabeli so predstavljeni podatki o dopingu na poletnih OI.
Sestavljena je iz 123 opazovanj in 7 spremenljivk:.
\begin{enumerate}
\item{\verb|Athlete| (imenska spremenljivka)}
\item{\verb|Sex| (imenska spremenljivka)}
\item{\verb|Country| (imenska spremenljivka)}
\item{\verb|Sport| (imenska spremenljivka)}
\item{\verb|Banned.substance| (imenska spremenljivka)}
\item{\verb|Place| (imenska spremenljivka)}
\item{\verb|Year| (številska spremenljivka)}

\end{enumerate}
\
\\
Tabela \verb|doping.ZOI|:
V program sem jo uvozila kot CSV datoteko. V tabeli so predstavljeni podatki o dopingu na zimskih OI.
Sestavljena je iz 22 opazovanj in 7 spremenljivk:.
\begin{enumerate}
\item{\verb|Athlete| (imenska spremenljivka)}
\item{\verb|Sex| (imenska spremenljivka)}
\item{\verb|Country| (imenska spremenljivka)}
\item{\verb|Sport| (imenska spremenljivka)}
\item{\verb|Banned.substance| (imenska spremenljivka)}
\item{\verb|Place| (imenska spremenljivka)}
\item{\verb|Year| (številska spremenljivka)}

\end{enumerate}

\
\\
Tabela \verb|bronze.medals|(odvzete bronaste medalje, zaradi dopinga):
V program sem jo uvozila kot html tabelo, iz spletene strani \url{http://en.wikipedia.org/wiki/List_of_stripped_Olympic_medals}
Sestavljena je iz 20 opazovanj in 3 spremenljivk:.
\begin{enumerate}
\item{\verb|Athlete| (imenska spremenljivka)}
\item{\verb|Country| (imenska spremenljivka)}
\item{\verb|Event| (imenska spremenljivka)}

\end{enumerate}

\
\\
Tabela \verb|gold.medals| (odvzete zlate medalje, zaradi dopinga):
V program sem jo uvozila kot html tabelo, iz spletene strani \url{http://en.wikipedia.org/wiki/List_of_stripped_Olympic_medals}
Sestavljena je iz 29 opazovanj in 3 spremenljivk:.
\begin{enumerate}
\item{\verb|Athlete| (imenska spremenljivka)}
\item{\verb|Country| (imenska spremenljivka)}
\item{\verb|Event| (imenska spremenljivka)}

\end{enumerate}

\pagebreak
\
\\
Tabela \verb|silver.medals| (odvzete srebrne medalje, zaradi dopinga):
V program sem jo uvozila kot html tabelo, iz spletene strani \url{http://en.wikipedia.org/wiki/List_of_stripped_Olympic_medals}
Sestavljena je iz 14 opazovanj in 3 spremenljivk:.
\begin{enumerate}
\item{\verb|Athlete| (imenska spremenljivka)}
\item{\verb|Country| (imenska spremenljivka)}
\item{\verb|Event| (imenska spremenljivka)}

\end{enumerate}
\
\\
Tabela \verb|CPI|:
V program sem jo uvozila kot CSV datoteko. V tabeli so predstavljeni indeksi korupcije v posameznih državah.
Sestavljena je iz 174 opazovanj in 2 spremenljivk:
\begin{enumerate}
\item{\verb|Country| (imenska spremenljivka)}
\item{\verb|CPI2014| (številska spremenljivka)}

\end{enumerate}
\
\\
Tabela \verb|technology|:
V program sem jo uvozila kot html tabelo, iz spletene strani \url{http://www.nationmaster.com/country-info/stats/Economy/Technology-index}.
V tabeli so predstavljeni tehnološki indeksi v posameznih državah.
Sestavljena je iz 109 opazovanj in 4 spremenljivk:
\begin{enumerate}

\item{\verb|#| (številska spremenljivka)}
\item{\verb|COUNTRY| (imenska spremenljivka)}
\item{\verb|AMOUNT| (imenska spremenljivka)}
\item{\verb|DATE| (številska spremenljivka)}
\end{enumerate}
\
\\
V vsaki izmed naštetih tabeli sem preverila tipe spremenljivk ter ga po potrebi spremenila. Odstranila sem nepotrebne znake, vrstice, stolpce in dodala nove.
\\




%\includepdf[pages={1-4}]{../slike/grafi.pdf}


\newpage

\section{Analiza in vizualizacija podatkov}

Na podlagi danih podatkov zbranih v tabelah sem narisala 8 grafov (6 stolpičnih, 2 tortna diagrama).
Program za risnje grafov sem napisala v datoteki \verb|grafi.r|, ki sem ga shranila v mapo \verb|slike|. Zaradi lažjega risanja grafov, sem na novo definirala nekaj spremenljivk, funkcij in pomožnih tabel.
\
\\
\begin{itemize}
\item{\verb|Graf 1|:}

\makebox[\textwidth][c]{
\includegraphics[width=\textwidth]{../slike/graf/1.pdf}
}
INTERPRETACIJA:
\\
\verb|Graf 1| prikazuje število dopinških primerov na posameznih zimskih Olimpijskih igrah. Očitno je, da izstopajo Olimpijske igre v Salt Lake Cityju - 2002 in igre v Sochiju - 2014.

\
\item{\verb|Graf 2|:}

\makebox[\textwidth][c]{
\includegraphics[width=\textwidth]{../slike/graf/2.pdf}
}
INTERPRETACIJA:
\\
\verb|Graf 2| prikazuje število dopinških primerov na posameznih poletnih Olimpijskih igrah. Očitno je, da izstopajo Olimpijske igre iz leta 2004, ki so potekale v Atenah.
\
\pagebreak
\item{\verb|Graf 3|:}

\makebox[\textwidth][c]{
\includegraphics[width=\textwidth]{../slike/graf/4.pdf}
}
INTERPRETACIJA:
\\
\verb|Graf 3| prikazuje število dopinških primerov na zimskih Olimpijskih igrah po posameznih športih. Največ dopinških škandalov (10) je bilo zabeleženih med tekači na smučeh, kar me ne preseneča, saj sem v medijih zasledila, da naj bi nekatere nacije v tej panogi celo načrtno izvajale doping.

\
\pagebreak
\item{\verb|Graf 4|:}

\makebox[\textwidth][c]{
\includegraphics[width=\textwidth]{../slike/graf/5.pdf}
}
INTERPRETACIJA:
\\
\verb|Graf 4| prikazuje število dopinških primerov na poletnih Olimpijskih igrah po posameznih športih. Daleč največ dopinških škandalov (39) je bilo zabeleženih v atletiki in v dvigovanju uteži. Razloge, da ti dve šprotni panogi kotirata tako visoko, gre iskati v naravi samega športa in v sistematičnem dopingu, ki je vsaj v atletiki vsem dobro znan že iz časov Vzhodne Nemčije.


\
\pagebreak
\item{\verb|Graf 5|:}

\makebox[\textwidth][c]{
\includegraphics[width=\textwidth]{../slike/graf/6.pdf}
}
INTERPRETACIJA:
\\
\verb|Graf 5| prikazuje razmerje dopinških primerov po spolu na zimskih Olimpijskih igrah.

\
\pagebreak
\item{\verb|Graf 6|:}

\makebox[\textwidth][c]{
\includegraphics[width=\textwidth]{../slike/graf/7.pdf}
}
INTERPRETACIJA:
\\
\verb|Graf 6| prikazuje razmerje dopinških primerov po spolu na poletnih Olimpijskih igrah.

\
\pagebreak
\item{\verb|Graf 7|:}

\makebox[\textwidth][c]{
\includegraphics[width=\textwidth]{../slike/graf/8.pdf}
}
INTERPRETACIJA:
\\
\verb|Graf 7| prikazuje število zlorab posameznih nedovoljenih sredstev. Razvidno je, da si športniki največkrat pomagajo z anaboličnimi steroidi, Furosemidom, Nandrolonom in Stanozololom.

\
\pagebreak
\item{\verb|Graf 8|:}

\makebox[\textwidth][c]{
\includegraphics[width=\textwidth]{../slike/graf/3.pdf}
}
INTERPRETACIJA:
\\
\verb|Graf 8| prikazuje število odvzetih olimpijskih medalj zaradi dopinga.
\end{itemize}

\
\pagebreak
\\
V tretji fazi pa sem uvožene podatke prikazala tudi na zemljevidu.
\\
\\
Program za risanje zemljevidov sem napisala v datoteki \verb|vizualizacija.r|, ki se nahaja v mapi \verb|vizualizacija|.
V ta namen sem v \verb|R| uvozila zemljevid sveta v obliki SHP.
\\
\\
Prva težava na katero sem naletela v tej fazi, je bila ta, da dveh držav, za kateri sem uvozila podatke ni bilo na zemljevidu, tako da ju nisem mogla prikazati.
\\
Preden sem se lotila risanja zemljevidov sem uredila podatke, tako da so se že prej uvoženi podatki ujemali z državami na zemljevidu sveta. Za ujemanje držav je bilo nekatere podatke potrebno tudi spremeniti oziroma nekoliko popraviti (npr.: Great Britain - United Kingdom, West Gemrany - Germany Soviet Union - Russia).
\\
\\
Na zemljevide sem dodala tudi nekaj oznak, vendar z njimi raje nisem pretiravala, saj bi s tem zemljevidi postali težje berljivi.
\
\pagebreak

Narisala sem 4 zemljevide:
\begin{itemize}

\item{\verb|Zemljevid 1|:}

\includegraphics[width=\textwidth]{../slike/zemljevidi/1.pdf}

INTERPRETACIJA:
\\
\verb|Zemljevid 1| prikazuje število dopingiranih športnikov iz posamezne države. Največ dopingiranih športnikov prihaja iz Rusije (13) in Združenih držav Amerike (12). Dani rezultati so me rahlo presenetili, saj vemo, da je bil v času Vzhodne Nemčije dopingiran praktično vsak športnik. 
\
\pagebreak
\item{\verb|Zemljevid 2|:}

\makebox[\textwidth][c]{
\includegraphics[width=\textwidth]{../slike/zemljevidi/2.pdf}
}
INTERPRETACIJA:
\\
\verb|Zemljevid 2| prikazuje, katero je najpogosteje uporabljeno nedovoljeno sredstvo v posamezni državi (imenska spremenljivka).
Ker je moja baza podtakov relativno majhna, sem pri risanju tega zemljevida upoštevala zgolj države, v katerih je bilo vsaj 6 dopinških primerov. Podatke za ostale države sem ocenila za nerelavantne.


\
\pagebreak
\item{\verb|Zemljevid 3|:}

\makebox[\textwidth][c]{
\includegraphics[width=\textwidth]{../slike/zemljevidi/3.pdf}
}
INTERPRETACIJA:
\\
\verb|Zemljevid 3| prikazuje število odvzetih medalj na Olimpijskih igrah, zaradi pozitivnega doping testa.

\
\pagebreak
\item{\verb|Zemljevid 4|:}

\makebox[\textwidth][c]{
\includegraphics[width=\textwidth]{../slike/zemljevidi/4.pdf}
}
INTERPRETACIJA:
\\
\verb|Zemljevid 4| prikazuje povprečno število dopinških primerov glede na državo, ki je gostila Olimpijske igre. 
Ker so nekatere države gostile Olimpijske igre večkrat, sem se odločila da bom vzela povprečje, s tem pa so tudi dani podatki postali bolj primerljivi.
\end{itemize}

\
\pagebreak

\section{Napredna analiza podatkov}

Ker je doping, zgolj ena izmed oblik korupcije, je bila moja prva ideja, da število dopinških primerv v posmeznih državah primerjam z njihovim indeksom korupcija. 
\\
Vendar pa, kot je iz spodnjega grafa očitno, je bila moja domneva napačna. Med indeksom korupcije in številom dopinških primerov ni nobene vidnejše povezave.
\\
\\
\makebox[\textwidth][c]{
\includegraphics[width=\textwidth]{../slike/napredna_analiza/1.pdf}
}

\
\\
Na podalgi zgornjega grafa sem opazila, da je doping najbolj razširjen med Evropskimi državami. Če izuzamemo Združene države Amerike, je zgolj v Evropski državah število dopinških primerov večje ali enako 4. Slednja ugotovitev pa je razvidna tudi iz dendrograma na naslednji strani.

\makebox[\textwidth][c]{
\includegraphics[width=\textwidth]{../slike/napredna_analiza/2.pdf}
}

\
\pagebreak
\\
V spodnjem grafu sem predstavila povezavo med tehnološkim indeksom in številom dopinških primerov. Vidimo da imajo praktično vse države sorazmerno visok tehnološki indeks, zgolj 5 držav ima indeks manjši od 3. Na tem grafu pa sem uporabila tudi funkcijo \verb|loess|. S pomočjo \verb|loess| modela sem prišla do ugotovitve, da ima večina držav, v katerih je prisoten doping, tehnološki indeks med 4 in 5.
\\
\\

\makebox[\textwidth][c]{
\includegraphics[width=\textwidth]{../slike/napredna_analiza/3.pdf}
}
\
\pagebreak
\\
Ker pa sem želela dobiti tudi širši vpogled v svet dopinga, sem za podatke o številu dopinških testov (od leta 2003 do leta 2010) narisala 3 krivulje, ki se najbolje prilegajo danim podatkom.
\


\makebox[\textwidth][c]{
\includegraphics[width=\textwidth]{../slike/napredna_analiza/4.pdf}
}
\
\pagebreak

\
Ter še zadnji graf, ki ponazarja število pozitivnih dopinških testov od leta 2003 do leta 2010. Vidimo, da je bilo največ pozitivnih vzorcev odvzetih med letoma 2008 in 2009.
\\
\\
\makebox[\textwidth][c]{
\includegraphics[width=\textwidth]{../slike/napredna_analiza/5.pdf}
}

\newpage
\section{Zaključek}
Moje mnenje je, da se bo doping v prihodnjih letih še bolj razširil, vendar ne zaradi napredka v tehnologiji ali zaradi lažjega dostopa do črnega trga, ampak zgolj zaradi človeškega pohlepa po uspehu.

\newpage
\begin{thebibliography}{4}
\bibitem{1}
  \url{http://en.wikipedia.org/wiki/Doping_at_the_Olympic_Games}\\
  {Accessed: 28-02-2015}
\bibitem{2}
  \url{http://en.wikipedia.org/wiki/List_of_doping_cases_in_athletics}\\
  {Accessed: 28-02-2015}
\bibitem{3}
  \url{http://en.wikipedia.org/wiki/List_of_stripped_Olympic_medals}\\
  {Accessed: 28-02-2015}
\bibitem{4}
  \url{www.huffingtonpost.com/2014/02/16/olympics-drug-testing-medals-stripped_n_4789565.html}\\
  {Accessed: 28-02-2015}
\bibitem{5}
  \url{http://en.wikipedia.org/wiki/List_of_doping_cases_in_sport}
 

\end{thebibliography}

\end{document}
